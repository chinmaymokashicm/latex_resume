%-------------------------
% Resume in Latex
% Author : Chinmay Mokashi
% License : MIT
%------------------------

\documentclass[a4paper,10.8pt]{article}

% \usepackage{mycolors}
\usepackage{latexsym}
\usepackage[empty]{fullpage}
\usepackage{titlesec}
\usepackage{marvosym}
\usepackage[usenames,dvipsnames]{color}
% \usepackage{mycolors}
\usepackage{verbatim}
\usepackage{enumitem}
\usepackage[pdftex]{hyperref}
\usepackage{fancyhdr}
\usepackage{array}
\usepackage{tabularx}


\pagestyle{fancy}
\fancyhf{} % clear all header and footer fields
\fancyfoot{}
\renewcommand{\headrulewidth}{0pt}
\renewcommand{\footrulewidth}{0pt}

% Adjust margins
\addtolength{\oddsidemargin}{-0.375in}
\addtolength{\evensidemargin}{-0.375in}
\addtolength{\textwidth}{1in}
\addtolength{\topmargin}{-.5in}
\addtolength{\textheight}{1in}

\urlstyle{rm}

\raggedbottom
\raggedright
\setlength{\tabcolsep}{0in}

% Sections formatting
\titleformat{\section}{
  \vspace{-3pt}\scshape\raggedright\large
}{}{0em}{}[\color{black}\titlerule \vspace{-5pt}]

%-------------------------
% Custom commands
\newcommand{\resumeItem}[2]{
  \item\small{
    \textbf{#1}{: #2 \vspace{-2pt}}
  }
}

\newcommand{\resumeItemWithoutTitle}[1]{
  \item\small{
    {\vspace{-2pt}}
  }
}

\newcommand{\resumeItemNested}[2]{
   \vspace{-1pt}\item
   \textbf{#1}
   #2
}

\newcommand{\resumeSubheading}[4]{
  \vspace{-1pt}\item
    \begin{tabular*}{0.97\textwidth}{l@{\extracolsep{\fill}}r}
      \textbf{#1} & #2 \\
      \textit{\small#3} & \textit{\small #4} \\
    \end{tabular*}\vspace{-5pt}
}



\newcommand{\resumeSubItem}[2]{\resumeItem{#1}{#2}\vspace{-4pt}}

\renewcommand{\labelitemii}{$\circ$}

\newcommand{\resumeSubHeadingListStart}{\begin{itemize}[leftmargin=*]}
\newcommand{\resumeSubHeadingListEnd}{\end{itemize}}
\newcommand{\resumeItemListStart}{\begin{itemize}}
\newcommand{\resumeItemListEnd}{\end{itemize}\vspace{-5pt}}

\newcommand{\tabitem}{\textbullet~~}

\hyphenpenalty=100000

%-------------------------------------------
%%%%%%  CV STARTS HERE  %%%%%%%%%%%%%%%%%%%%%%%%%%%%


\begin{document}

%----------HEADING-----------------
\begin{tabular*}{\textwidth}{l@{\extracolsep{\fill}}r}
  \textbf{{\LARGE Chinmay Mokashi}} & Email : \href{mailto:chinmaymokashicm@gmail.com}{chinmaymokashicm@gmail.com}\\
  \href{https://www.linkedin.com/in/chinmay-mokashi/}{Linkedin: https://www.linkedin.com/in/chinmay-mokashi/} & Mobile : +1-346-339-7614 \\
  \href{https://github.com/chinmaymokashicm}{Github: https://github.com/chinmaymokashicm} & \href{https://chinmaymokashi.page}{Personal Website: https://chinmaymokashi.page} \\
\end{tabular*}

%
%--------SKILLS------------
\section{Skills Summary}
	\resumeSubHeadingListStart
	\resumeSubItem{Data Analysis and Machine Learning}{SQL, Matplotlib, Pytorch, scikitlearn, scikit-image, BERT, Tableau}
	\resumeSubItem{Data Engineering}{Google Cloud, MySQL, REST API, MongoDB, PySpark, BeautifulSoup, NLTK}
	\resumeSubItem{Software Engineering}{Python, Node, React, Flask, Ansible}
  \resumeSubItem{Biomedical Data}{DICOM, Epic EHR, HL7, FHIR, RxNORM}
\resumeSubHeadingListEnd

%-----------EXPERIENCE-----------------
\section{Experience}
  \resumeSubHeadingListStart
    \resumeSubheading
    {School of Biomedical Informatics, UTHealth}{Houston, TX}
    {Graduate Researcher}{January 2021 - May 2023}
    \begin{itemize}
        \item Built separate automatic deep learning tools to screen patients for endometriosis and inverted papilloma\\
        \textit{Researched and implemented \textbf{U-Net (CNN) based nnU-Net pipelines} on annotated 3D MRI sequences and CT images for semantic image segmentation
        (\textbf{Average Dice score - 0.75}).
        \textbf{Collaborated with healthcare professionals} to review, annotate/segment and clean MRI data. 
        Understood medical requirements in preparing datasets.
        Performed text processing on radiology reports to extract key information such as MRNs and endometriosis status.
        Analyzed deep learning outcomes and created reports.
        \\
        }
        \item Built a visualization tool to identify misinformation on YouTube regarding vaping\\
        \textit{Created a data pipeline to analyze \textbf{Youtube comments}.
        Built a Flask application to label Youtube videos into predefined categories.
        Used the \textbf{Youtube Data API} to collect, de-identify and analyze data and looked for trends.
        Looked for \textbf{misinformation} using supervized and unsupervized (\textbf{topic modeling}) ML techniques.
        Presented results using Streamlit dashboard.}
      \end{itemize}

    \resumeSubheading
    {Harris Health System}{Houston, TX}
    {Data Analyst - Practicum}{January 2023 - May 2023}
    \begin{itemize}
        \item To assess the diagnostic characteristics of the Epic Sepsis Predictive Model (ESPM) \\
        Writing \textbf{complex SQL queries} on Oracle databases to derive computations and generate information to be used to analysis.
        Analyzing occurrence of sepsis during hospitalization or emergency visits.
    \end{itemize}

    \resumeSubheading
    {School of Biomedical Informatics, UTHealth}{Houston, TX}
    {Researcher - Directed Study}{January 2021 - May 2023}
    \begin{itemize}
        \item Extracting genomic and protein-protein interaction information from clinical trials text data\\
        \textit{Building \textbf{BERT-based} machine learning models to extract \textbf{genomic and protein-protein interaction information} from annotated BioCreative datasets.
        Applying these models to extract information of same from \textbf{clinical trials reports.}} 
        Performing gene normalization by matching gene mentions with their appropriate entrez identifiers, by generating 2.5 million unique mention-ID pairs and applying data processing and machine learning techniques.
        \item Built a purely client-side \href{https://retina.chinmaymokashi.page}{\underline{web-application}} to predict the risk of acute stroke in patients using OCTA images\\
        \textit{Built a frontend web-application with \textbf{image processing and vascular metric calculation} capabilities on \textbf{React.JS}. 
        Trained a ML classifier on OCTA images and deployed on the user interface (\textbf{AUC - 0.81}).}
    \end{itemize}

    \resumeSubheading
    {School of Biomedical Informatics, UTHealth}{Houston, TX}
    {Teaching Assistant}{January 2023 - May 2023}
    \begin{itemize}
      \item Updating class projects to be implemented on Orange Data Mining\\
      \textit{Updating projects in 'BMI 6323: Machine Learning in Biomedical Informatics' to be implemented in Orange Data Mining.
      Class for beginner level machine learning enthusiasts requiring no programming knowledge.
      }
    \end{itemize}

    \resumeSubheading
    {iOPEX Technologies}{Bangalore, India}
    {Senior Software Engineer}{May 2019 - December 2020}
    \begin{itemize}
      \item Built a custom email classifier for a major British telecommunications company\\
      \textit{\textbf{Led a team of 3 in data engineering design}, primarily based on \textbf{Google Cloud Platform}.
      \textbf{Built ETL pipelines} to clean, structure and train user emails to create custom email categories.
      Used logistic regression and SVM to \textbf{categorize emails into 10 non-exclusive labels} (\textbf{best AUC - 0.84}).
      \item Performed data analysis tasks on ad-hoc basis\\
      \textit{Worked on time-sensitive ad-hoc data analysis projects, such as insurance and time series transactions data.
      }
      }
    \end{itemize}
      
    \resumeSubheading
    {AthenasOwl, Quantiphi Inc.}{Mumbai, India}
    {Data Engineer}{January 2018 - January 2019}
    \begin{itemize}
        \item Ideated and developed data infrastructure and architecture of the AthenasOwl ML platform\\
        \textit{Built a \textbf{Gateway API} for authorization, authentication and user management of a multi-tenant \textbf{platform-as-a-service} (PaaS) infrastructure. 
        Leveraged GCP services to build a create-and-destroy architecture based Machine Learning solution. 
        Researched, designed and implemented a custom ML platform on a \textbf{Kubernetes} cluster using docker APIs.
        Collaborated with UI/UX vendor to create PaaS platform interface.}
    \end{itemize}

\resumeSubHeadingListEnd

% -----------PROJECTS-----------------
\section{Personal Projects}
\resumeSubHeadingListStart
\resumeSubItem{X-ray image classification to predict cardiomegaly}{Using image processing and transfer learning, constructed \textbf{10 different data pipelines} with different approaches and compared them to identify the best approach. Wrote a \href{https://github.com/chinmaymokashicm/bmi_6331/blob/e551df3264bb96c0502e67044fd2a4ad3c010afb/data_challenge/G1_data_challenge_report-1.pdf}{\underline{paper}} to report the results.}
\resumeSubItem{Interactive patient dashboard}{Using a large synthetic \textbf{NoSQL} database, constructed a \href{https://github.com/chinmaymokashicm/patient_dashboard}{\underline{dashboard}} for patient information with querying and visualization capabilities. Used MongoDB, React.JS, Node.JS and Recharts.JS.}
\resumeSubItem{Authentication and Authorization API}{Designed and built a gateway \href{https://github.com/chinsisme/gateway-router}{\underline{router}} for authentication and authorization of user requests on to a PaaS cloud platform. Used Node.JS, GCP services and Kubernetes.}
\resumeSubHeadingListEnd

%-----------EDUCATION-----------------
\section{Education}
  \resumeSubHeadingListStart
    \resumeSubheading
      {University of Texas Health Science Center at Houston (UTHealth)}{Houston, TX}
      {Master of Science in Biomedical Informatics}{January 2021 - May 2023}
      
	   {\scriptsize \textit{Courses: Medical Imaging and Signal Pattern Recognition, Big Data, Machine Learning, Statistics in Biomedical Informatics}}
	    
    \resumeSubheading
      {National Institute of Technology, Rourkela}{Rourkela, India}
      {Bachelor of Technology in Biomedical Engineering}{August 2012 - May 2017}
  \resumeSubHeadingListEnd

%-----------Awards-----------------
% \section{Honors and Awards}
% \begin{description}[font=$\bullet$]
    % \item {Selected in top 20 students for the Code House event organized by VMware in August15 - August17, 2016.} 
% \item {Ranked first among batch of 60 students in my Computer Science Engineering Branch.}
% \item {Ranked fifth among batch of 500 students at High School Level A.I.S.S.E 2005}
% \end{description}
%-----------Certifications-----------------
\section{Certifications}
\resumeSubHeadingListStart
\resumeSubheading{Google Cloud Professional Data Engineer}{Worldwide}{\href{https://cloud.google.com/certification/data-engineer}{https://cloud.google.com/certification/data-engineer}}{September 2018 - September 2020}
\resumeSubHeadingListEnd


%-------------------------------------------
\end{document}